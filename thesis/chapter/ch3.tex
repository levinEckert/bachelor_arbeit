\chapter{Analyse der Ist-Situation und Anforderungsanalyse}
\label{ch:analyse}

%================================================================================================
%       3.1 Untersuchungsobjekt
%================================================================================================
\section{Analyse des Untersuchungsobjekts}
\label{sec:maschine}
\begin{itemize}
    \item Technischer Aufbau der betrachteten Maschine
    \item Identifikation kritischer Komponenten für die Schwingungsüberwachung
    \item Sensoren, wo usw.
\end{itemize}

%================================================================================================
%       3.2 Datenbasis und Infrastruktur
%================================================================================================
\section{Analyse der Datenbasis und Infrastruktur}
\label{sec:infrastruktur}

\subsection{Bereitstellung und Zugriff}
\begin{itemize}
    \item Schnittstellen zur Datenquelle (z. B. Datenbank, Cloud-Anbindung)
    \item Vorhandene IT-Infrastruktur am Standort
\end{itemize}

\subsection{Beschreibung der Rohdaten}
\begin{itemize}
    \item Datenformate und Datentypen (z. B. CSV, JSON)
    \item Zeitstempel-Synchronisation und Abtastraten
\end{itemize}

\subsection{Herausforderung Datenhistorie}
\begin{itemize}
    \item Status der verfügbaren historischen Fehlerdaten
    \item Begründung des gewählten Ansatzes bei geringer Datenbasis
\end{itemize}

%================================================================================================
%       3.3 Anforderungskatalog
%================================================================================================
\section{Anforderungsanalyse}
\label{sec:anforderungskatalog}
\begin{itemize}
    \item Tabellarische Aufstellung der Systemanforderungen
    \item Klassifizierung nach Muss-, Soll- und Kann-Kriterien
\end{itemize}