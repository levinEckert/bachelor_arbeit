\chapter{Evaluation und Ergebnisse}
\label{ch:evaluation}

%================================================================================================
%       6.1 Testdurchführung
%================================================================================================
\section{Testdurchführung}
\label{sec:testdurchfuehrung}
\begin{itemize}
    \item Beschreibung der gewählten Testszenarien
    \item Dokumentation der Versuchsbedingungen
\end{itemize}

%================================================================================================
%       6.2 Darstellung der Ergebnisse
%================================================================================================
\section{Darstellung der Ergebnisse}
\label{sec:ergebnisse_darstellung}

\subsection{Analyse der Schwingungsspektren}
\begin{itemize}
    \item Auswertung der gemessenen Frequenzspektren
    \item Interpretation der identifizierten Amplituden und Peaks
\end{itemize}

\subsection{Validierung des Algorithmus}
\begin{itemize}
    \item Überprüfung der Funktionsweise gegenüber der Zielsetzung
    \item Genauigkeit der Merkmalsextraktion
\end{itemize}

%================================================================================================
%       6.3 Soll-Ist-Vergleich
%================================================================================================
\section{Soll-Ist-Vergleich anhand des Anforderungskatalogs}
\label{sec:soll_ist_vergleich}
\begin{itemize}
    \item Abgleich der Ergebnisse mit der Anforderungstabelle aus Kap. 3.3
    \item Status der Umsetzung (Muss-, Soll- und Kann-Kriterien)
\end{itemize}

%================================================================================================
%       6.4 Diskussion der Datenqualität
%================================================================================================
\section{Diskussion der Datenqualität}
\label{sec:diskussion_daten}
\begin{itemize}
    \item Bewertung der Aussagekraft der vorhandenen Datenbasis
    \item Rückbezug auf die Problematik geringer historischer Fehlerdaten
\end{itemize}