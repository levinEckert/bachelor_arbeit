\chapter{Stand der Technik/theoretische Grundlagen/Literaturstudium}
\label{ch:grundlagen}

%================================================================================================
%       2.1 Instandhaltungsstrategien
%================================================================================================
\section{Instandhaltungsstrategien}
\label{sec:instandhaltung}
\begin{itemize}
    \item Abgrenzung: Reaktiv, Präventiv, Predictive Maintenance
\end{itemize}

%================================================================================================
%       2.2 Schwingungsanalyse
%================================================================================================
\section{Grundlagen der Schwingungsanalyse}
\label{sec:schwingungsanalyse}
\begin{itemize}
    \item Physikalische Größen (Beschleunigung, Frequenz)
    \item Typische Fehlermuster (Unwucht, Lager- und Ausrichtfehler) für Bauteile
\end{itemize}

%================================================================================================
%       2.3 Digitale Signalverarbeitung
%================================================================================================
\section{Digitale Signalverarbeitung}
\label{sec:signalverarbeitung}
\begin{itemize}
    \item Abtasttheorem und Aliasing
    \item Fast Fourier Transformation (FFT)
\end{itemize}

%================================================================================================
%       2.4 Kennwertanalyse
%================================================================================================
\section{Merkmalsextraktion und Kennwerte}
\label{sec:kennwerte}
\begin{itemize}
    \item RMS, Crest-Faktor, Kurtosis
\end{itemize}