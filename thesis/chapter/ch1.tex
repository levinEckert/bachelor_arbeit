\chapter{Einleitung}
\label{ch:einleitung}

% Einleitungstext: Hier kurz das Thema anreißen (Predictive Maintenance an Wuchtanlage)
% und den Leser abholen, bevor es in die Details geht.


%================================================================================================
%       1.1 Umfeld
%================================================================================================
\section{Umfeld}
\label{sec:umfeld}

Dieses Kapitel beschreibt den organisatorischen und technischen Kontext der Arbeit. Es stellt das Unternehmen \gls{SE}, den spezifischen Standort sowie das zu untersuchende Produkt und die Abteilung vor.

%------------------------------------------------------------------------------------------------
\subsection{Siemens Energy}
\label{subsec:siemens-energy}
%------------------------------------------------------------------------------------------------
\subsubsection{Standort Huttenstraße}
\label{subsubsec:standort-huttenstrae}
%------------------------------------------------------------------------------------------------
\subsubsection{Abteilung Maschinen Instandhaltung}
\label{subsubsec:abteilung-maschinen-instandhaltung}
%------------------------------------------------------------------------------------------------
\subsection{Gasturbinen (Produkt)}
\label{subsec:gasturbinen}
%================================================================================================
%       1.2 Motivation
%================================================================================================
\section{Motivation}
\label{sec:motivation}
In der industriellen Instandhaltung stellt der zuverlässige und wirtschaftliche Betrieb technischer Anlagen eine zentrale Anforderung dar.
Insbesondere bei Energieerzeugungsanlagen, wie z.B. Gasturbinen, führen ungeplante Ausfälle zu erheblichen Beeinträchtigungen der Produktionsprozesse sowie zu signifikanten Kosten- und Sicherheitsrisiken. Herkömmliche Wartungsstrategien basieren überwiegend auf festen Zeitintervallen, ohne den tatsächlichen Verschleißzustand der Komponenten zu berücksichtigen.

    Diese intervallorientierte Vorgehensweise bringt zwei wesentliche Probleme mit sich: Einerseits werden Wartungen an technisch intakten Anlagen durchgeführt, was zu unnötigen Stillstandzeiten und Ressourcenverschwendung führt.
    Andererseits können sich kritische Schäden zwischen den Wartungsintervallen unbemerkt entwickeln, was im schlimmsten Fall zu ungeplanten Ausfällen führt.
    Beide Szenarien verursachen erhebliche wirtschaftliche Verluste und beeinträchtigen die Effizienz der Instandhaltungsorganisation.

    Predictive Maintenance optimiert diesen Ablauf durch kontinuierliche Zustandsüberwachung anhand von Sensorik.
    Physikalische Messgrößen wie Schwingungen, Temperaturen oder elektrische Parameter werden kontinuierlich erfasst, verarbeitet und ausgewertet, um Abweichungen vom Normalzustand frühzeitig zu erkennen.
    Dieser datengetriebene Ansatz ermöglicht eine bedarfsgerechte Planung von Wartungsmaßnahmen und verschiebt den Fokus von reaktiver zu proaktiver Instandhaltung.

    Trotz des erkennbaren Potenzials ist die Implementierung von Predictive-Maintenance-Systemen in der Praxis noch begrenzt.
    Die notwendige Sensorinfrastruktur ist kostenintensiv und wird erst seit kurzer Zeit in den Anlagen integriert.
    Zudem fehlen häufig Fachkräfte mit spezifischen Kompetenzen in Instandhaltungstechnik, Sensorik und Datenanalyse.



    Die Arbeit entwickelt einen funktionsfähigen Ansatz zur datenbasierten Zustandsüberwachung und Ausfallvorhersage.
    Im Fokus steht die praktische Implementierung eines Algorithmus, der aus Sensordaten belastbare Prognosen über den Maschinenzustand und potenzielle Ausfallzeitpunkte berechnet.
%================================================================================================
%       1.3 Anforderungen
%================================================================================================
\section{Anforderungen}
\label{sec:anforderungen}

