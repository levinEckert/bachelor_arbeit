\chapter{Einleitung}
\label{ch:einleitung}
% 2-4 Seiten Turbinen Beschreibung und Einführung in Siemens Energy



%================================================================================================
%       Umfeld
%================================================================================================
\section{Umfeld}
\label{sec:umfeld}

%================================================================================================
\subsection{Siemens Energy}
\label{subsec:siemens-energy}
\gls{SE} AG ist ein deutsches, weltweit tätiges Energietechnologieunternehmen mit mehr als 100.000 Mitarbeitenden in über 90 Ländern.
Das Unternehmen ist auf Lösungen, Produkte und Dienstleistungen entlang nahezu der gesamten Energiewertschöpfungskette spezialisiert.
Dazu zählen unter anderem die Energieerzeugung (z.\,B. durch Gasturbinen, Dampfturbinen und erneuerbare Energien), der Energietransport (z.\,B. Hochspannungs-Gleichstrom-Übertragung, Transformatoren und Netztechnik) sowie Energiespeicherung und Netzstabilisierung.
Es entstand im Jahr 2020 als eigenständige, börsennotierte Gesellschaft durch die Abspaltung der früheren Gas- und Power-Division von Siemens AG\@.\cite{SiemensEnergyAbout2026}


%================================================================================================
\subsubsection{Standort Huttenstraße}
\label{subsubsec:standort-huttenstrae}
Der Standort Huttenstraße 12, 10553 in Berlin (Ortsteil Moabit) wird vom Unternehmen als „Gasturbinenwerk“ am Standort EnergySphere Berlin geführt.
Im Standortkontext wird EnergySphere Berlin als Verbund mehrerer Einrichtungen in der Region Berlin-Brandenburg beschrieben (u. a. mit Innovations-/Technologiebezug).\cite{siemensenergy_energysphere_berlin}
%Historie wahrscheinlich egal
% auch vorstellung von Corporate/Innovation Center, Silyzer




%================================================================================================
\subsubsection{Abteilung Maschinen Instandhaltung}
\label{subsubsec:abteilung-maschinen-instandhaltung}


\subsection{Gasturbinen}
\label{subsec:gasturbinen}
\acrolong{SE}
2-600 MW, am Standort Huttenstraße nur \enquote{Heavy Duty} Gasturbinen (120-600MW)



%================================================================================================
%       Ziel
%================================================================================================
\section{Ziel}
\label{sec:ziel}
Entwurf eines Condition Monitoring Systems für Motor und Getriebe einer Turbinenläufer Wuchtanlage.
Hohe Aussage Kraft und Sicherheit um Arbeit bei der Wartung zu reduzieren und ungeplante Stillstände zu minimieren.


%================================================================================================
%       Anforderungen
%================================================================================================
\subsection{Anforderungen}
\label{subsec:anforderungen}

\begin{table}[h]
    \centering
    \caption{Anforderungstabelle}
    \label{tab:anforderungen}
    \begin{tabularx}{\textwidth}{l X c}
        \toprule
        \textbf{ID} & \textbf{Anforderung} & \textbf{Priorität} \\
        \midrule
        A1 & X & MUSS \\
        A2 & X & SOLL \\
        A3 & X & KANN \\
        \bottomrule
    \end{tabularx}
\end{table}







